%Document details
\documentclass[titlepage, table]{article}
\title{Asset Allocation - Investment Advice to Best Pensions}    %Fag og opgave nr.
\author{Anders Bang and Andreas Poulsen (fjr422, zmh741)}
\date{\today}
%Dato

%Packages
\usepackage[utf8]{inputenc}
\usepackage[autostyle=true]{csquotes}
\usepackage{amsmath,amssymb,amsthm}
\usepackage{bbm}
\usepackage[danish, english]{babel}
\usepackage{pdfpages}
\usepackage{graphicx}
\usepackage{wrapfig}
\usepackage{dsfont}
\usepackage{fancyhdr}
\usepackage{mathtools}
\usepackage{lastpage}
\usepackage{hyperref} % Skal man kunne klikke på referencer?
\usepackage{caption}
\usepackage{lipsum}
\usepackage{multirow}
\usepackage[export]{adjustbox}
\usepackage{physics}
\usepackage{algpseudocode} %pseudocode
\usepackage{algorithm}
\usepackage{adjustbox} % to rotate tables

%Set up the page
\usepackage[top=2.5 cm, bottom= 2.5 cm, textwidth=480 pt]{geometry}
%\usepackage[a4paper, top=2.5, bottom=2.5]{geometry}
 %\geometry{
 %a4paper,
 %total={170mm,257mm},
 %left=20mm,
 %top=20mm
 %}
 \linespread{1.15}

\pagestyle{fancy}
\fancypagestyle{plain}{}
\fancyhf{}

\renewcommand{\footrulewidth}{0.4pt} %Linje i bunden?

%\setlength{\headheight}{30pt}  % Increase the header height
%\setlength{\headsep}{25pt}  

%Header and footer
\rhead{\today}                      %Dato
\chead{Asset Allocation - Investment Advice to Best Pensions}                %Fag og nr.
\lhead{fjr422, zmh741}                    %Forfatter
\rfoot{Page \textbf{\thepage}}

%Declaring commands
%\DeclareMathSymbol{*}{\mathbin}{symbols}{"01}

\newcommand{\indep}{\perp \!\!\! \perp} %Independent

\newcommand{\N}{\mathbb{N}} %Natural numbers
\newcommand{\E}{\mathbb{E}} %Expectation E
\newcommand{\A}{\mathcal{A}} %Caligrafic A (Inf. Gen.)
\newcommand{\G}{\mathcal{G}} 
\newcommand{\F}{\mathcal{F}}
\newcommand{\R}{\mathbb{R}} %Real numbers
\newcommand{\C}{\mathbb{C}} %Complex numbers
\renewcommand{\P}{\mathbb{P}} %Probability P
\renewcommand{\L}{\mathcal{L}}
\renewcommand{\E}[1]{\mathbb{E}\left[ #1 \right]} %Expectation
\newcommand{\V}{\mathcal{V}}
\newcommand{\Q}{\mathbb{Q}}
\newcommand{\cov}[1]{\text{Cov}\left( #1 \right)} %covariance
\renewcommand{\var}[1]{\text{Var}\left( #1 \right)} %Variance
\newcommand{\inprod}[1]{\left\langle #1 \right\rangle} %Inner product
\newcommand{\assim}{\overset{a.s.}{\sim}} %asymptotic similar
\newcommand{\normal}[2]{\mathcal{N}\left( #1 , #2  \right)} %Normal Distribution
\newcommand{\asto}{\overset{a.s.}{\to}} %Converge almost-surely
\newcommand{\pto}{\overset{\mathbb{P}}{\to}} %Converge in Probability
\newcommand{\dto}{\overset{d}{\to}} %Converge in distribution
\newcommand{\T}{\top} %Transpose-sign
\renewcommand{\norm}[1]{\left\lvert\left\lvert #1 \right\rvert \right\rvert}  %Norm
\renewcommand{\abs}[1]{\left\lvert #1 \right\rvert} %Absoulute Value
\newcommand{\set}[1]{\left\{ #1 \right\}} %For sets
\newcommand{\ind}[1]{\mathds{1}_{\set{#1}}} %Indicator
\newcommand{\EQC}[2]{\mathbb{E}^{\mathbb{Q}} \left[ #1 \mid #2\right]}
\newcommand\numberthis{\addtocounter{equation}{1}\tag{\theequation}} %Adding to equation counter for general enviornment
\newcommand{\pa}[1]{\left( #1 \right)}
\DeclareMathOperator{\VaRa}{\text{VaR}_\alpha}
%\setcounter{section}{0}         %Sets the numbering of sections.
\setcounter{secnumdepth}{0} %Removes the numbers from the sections.
\setlength{\parindent}{0pt}

%For at kunne læse rmd output
%\input{RMD_format.tex}

\begin{document}
\maketitle
\newpage
\tableofcontents
\newpage

\section{Introduction}
This report perform an analysis of the of the asset allocation for Best Pensions and provides recommendations as to how the allocation strategy might be improved in order to provide even higher returns for the clients of Best Pension.

The product design in Best Pension currently reflects the high degree of dependence towards providing the promised guarantees at termination of the Target Date Funds (TDF).
Based on the analysis it is strongly recommended to revise the balance between allocation to an active fund and TDF's.
\textcolor{red}{In particular we recommend that Best Pension decreases $L_{target}$ to and converts to a CPPI strategy with multiply of $m$. Further we recommend that Best Pension allows for investments in foreign markets for the active portfolio.}
Further it is advisable to continuously monitor performance in order to maintain a satisfactory level of performance.

In the analysis period the majority of active assets have outperformed investments in zero-coupon-bonds (ZC-bonds) over a 10 year horizon. This is observed in spite of some major down periods for the active assets. However the active in general yield higher returns and are therefore able to recover.
In particular it is observed that the American long-assets have outperformed equivelant European assets. Expanding the investable asset class to allow for investing in foreign markets would therfore significantly have benefited to increase the guarantees of Best Pensions clients.
Based on the analysis we recommend that for the active fund Best Pension invests in the following assets: \textcolor{red}{assets}. The allocation should be based on an \textcolor{red}{strategy}.

Since it is observed that the active assets outperform ZC-bonds an allocation where a higher degree of the initial contribution towards the active fund will in general lead to a larger terminal guarantee for the TDF. As such reducing $L_{target}$ is advisable.
This has become even more relevant as the period of \textit{Quantitative Easening} seems to be over. This means that the original guarantees can be honoured with a much lower initial allocation towards ZC-bonds.
This will in turn mean a higher allocation to the active fund which should yield a higher return and thus a higher terminal guarantee. \newline



This report provides an extensive analysis of allocation strategies for Best Pensions, 
a European pension provider with a strong belief in guarantees. 
Best Pensions offer a series of target date funds, which guarantee the cliets to at 
least get their invested money back.

Due to recent critique of the company's investment strategy, especially
their low allocation towards equity, Best Pensions want to explore more modern approaches
to asset allocation. But they are not willing to deviate from their current product design.

The goal of this report is thus to provide investment advices, which aligns with modern
portfolio theory, while still ensuring that Best Pensions can honour their guarantees.


\section{Recommendations regarding Best Pensions' allocation in their Active Portfolio}
In the following section, we will provide an extensive analysis of the current stock market situation. 
Based on this analysis, we will present our recommendations on, how Best Pensions' should allocate the assets in their Active Portfolio.

\subsection{Comparative Analysis of Factor Models}

The Executive Board of Best Pensions has expressed interest in using a more modern factor approach in their asset allocation. 
This approach was originally introduced by Fama and French in 1993, and risk factors
have since become a central component of portfolio theory. 

The market that Fama and French considered in their original paper was the US stock market from 1963 to 1991, thus we 
start by showing that the factor model approach is still relevant in the current market situation, 
and that the approach also applies to the European market.
\newline

The Executive Board has shown particular interest in the market factor, the momentum factor, and the size factor.
The market factor, MKT, suggests that the overall market returns
affects the returns of individual assets. The momentum factor, MOM, suggests that
assets which performed well in the past will continue to perform well in the future.
And the size factor, SMB, suggests that the size of a company affects its returns,
with smaller companies tending to outperform larger companies.
\newline

To ensure that the original analysis is not outdated, we perform a comparative 
analysis of the three factors for the period from September 2004 to December 2024. 
During this analysis we adopt the methodology developed in the original factor analysis by Fama and French.

Moreover, to ensure that the factor approach is relevant to a European investor, 
we analyze the relevancy of the three factors in three different market/investor settings: 

\begin{itemize}
    \item A USD-based investor investing in US Equity.
    \item A EUR-based investor investing in US Equity. 
    \item A EUR-based investor investing in European Equity. 
\end{itemize}

The results of our analysis shows that the three factors are indeed still relevant in all three settings. 
Since Best Pensions is a EUR-based investor investing in possibly both US and European Equities, 
it is indeed suitable for Best Pensions to consider these factors in their asset allocation.
\newline

\subsubsection{Methodology}
To form our proxies for the risk factors, we divide the markets into 6 portfolios based on size and momentum of the stocks, and use the monthly returns of the portfolios.

The size factor,SMB , is calculated by taking the average return of the 3 portfolios containing the samllest firms and 
subtracting the average return of the 3 portfolios containing the largest firms at the end of each month.
The momentum factor, MOM, is calculated by taking the average return of the 2 highest performing portfolios
and subtracting the average return of the 2 worst performing portfolios.
The market factor, RM-RF, is calculated as the excess return of the value-weigted market portfolio over the risk-free rate.
\newline 

We then divide our markets into 25 portfolios based on five size quintiles and five momentum quintiles,
and perform three linear regressions with the following form to each of the portfolios in the 3 market/investor settings:

\begin{align*}
    &\text{1.} \hspace{2cm} R(t)-RF(t) = a + b (RM(t)-RF(t)) + e(t) \\
    &\text{2.} \hspace{2cm} R(t)-RF(t) = a + s SMB(t) + m MOM(t) + e(t) \\
    &\text{3.} \hspace{2cm} R(t)-RF(t) = a + b (RM(t)-RF(t)) + s SMB(t) + m MOM(t) + e(t) \\
\end{align*}

Since the data used in this analysis is from the Kenneth R. French Data Library, 
all of the returns are calculated in USD. 

When considering a EUR-based investors, we therefore need to adjust the returns 
to account for currency loss or gains, by including the return of the USD/EUR-exchange rate. 

\subsubsection{USD-based investor investing in US Equity}
In Table \ref{R2_USD_US} the adjusted $R^2$ values from the three regressions are shown.
From the table we note, that the adjusted $R^2$ values increase significantly when
including all three factors, indicating that the three-factor model explains the variation in bond returns better 
than both the MKT only model and the SMB-MOM model. Thus we concentrate on the three-factor model.
% latex table generated in R 4.4.1 by xtable 1.8-4 package
% Tue Oct 21 07:56:15 2025
\begin{table}[ht]
\centering
\begin{tabular}{lccccc|ccccc|ccccc}
  \hline
    & \multicolumn{5}{c|}{MKT only} & \multicolumn{5}{c|}{SMB and MOM only} & \multicolumn{5}{c}{All three factors} \\ 
 & Low & 2 & 3 & 4 & High & Low & 2 & 3 & 4 & High & Low & 2 & 3 & 4 & High \\ 
  \hline
Small & 0.66 & 0.71 & 0.73 & 0.69 & 0.66 & 0.69 & 0.65 & 0.58 & 0.54 & 0.52 & 0.93 & 0.95 & 0.93 & 0.90 & 0.91 \\ 
  2 & 0.69 & 0.77 & 0.77 & 0.77 & 0.73 & 0.71 & 0.62 & 0.56 & 0.52 & 0.44 & 0.96 & 0.96 & 0.95 & 0.94 & 0.93 \\ 
  3 & 0.70 & 0.81 & 0.84 & 0.83 & 0.76 & 0.65 & 0.55 & 0.49 & 0.45 & 0.36 & 0.93 & 0.95 & 0.95 & 0.94 & 0.93 \\ 
  4 & 0.69 & 0.84 & 0.89 & 0.89 & 0.76 & 0.61 & 0.48 & 0.37 & 0.26 & 0.22 & 0.90 & 0.94 & 0.93 & 0.93 & 0.88 \\ 
  Big & 0.67 & 0.80 & 0.91 & 0.88 & 0.75 & 0.51 & 0.45 & 0.26 & 0.08 & 0.05 & 0.86 & 0.93 & 0.93 & 0.91 & 0.88 \\ 
   \hline
\end{tabular}
\caption{Adjusted $R^2$ values from the three regressions performed on the 25 portfolios formed on size and momentum, for the market of an USD-based investor investing in US Equity. }
\label{R2_USD_US}
\end{table}

\newline

In Table \ref{Factors_USD_US} the regression estimates from the three-factor model 
for a USD-based investor investing in US Equity is shown. 
In the colored sections, we have the slopes of the size factor (SMB) 
and the momentum factor (MOM) for each of the 25 portfolios.

We note that for any fixed momentum quintile,
the slope of the size factor is decreasing as we move from 
the portfolios with the smallest sized assets to the portfolios with 
the largest sized asset.

Similarly, for any fixed size quintile, the slope of the momentum factor is increasing as we move from
the portfolios with the lowest momentum to the portfolios with the highest momentum. 

From the associated t-statistics, we see that in a single testing framework the factors are statistically significant
in almost all the portfolios. 

Moreover, we note that the adjusted R-squared values are all very high, meaning that 
the model explains a lot of the common variation in the portfolio returns.

From this we conclude that the three-factor model is indeed still relevant 
for a USD-based investor investing in US Equity in the current market.

\input{R/Output/factors_USD_US_colored_format.tex}


\newpage
\subsubsection{EUR-based investor investing in US Equity and EU Equity}
An EUR-based investor investing in US equities will both be subject to the returns
of the equities and the returns of the currency exchange rate between USD and EUR.

Moreover, the data from the Kenneth R. French Data Library is all in USD. So the returns on the EU Equities
has been added some exchange rate gain/loss, which we need to remove. 

To account for the exchange rate returns, we transform the USD returns from the Data Library into EUR returns using the 
exchange rate returns from the European Central Bank.
\newline

In Table \ref{R2_EUR} the adjusted $R^2$ values from the three regressions are shown for an EUR-based investor investing in both US Equity and EU Equity.
We again note, that the adjusted $R^2$ values increase significantly when
including all three factors, indicating that the three-factor model explains the common variation in the stock returns better than the other two models. 
We also note, that the adjusted $R^2$ values are close to 1 in most of the portfolios for the three-factor model, yielding that the model is very good at explaining the returns. 
Thus we again concentrate on the three-factor model.
\newline

% latex table generated in R 4.4.1 by xtable 1.8-4 package
% Tue Oct 21 08:41:05 2025
\begin{table}[ht]
\centering
\begin{tabular}{lccccc|ccccc|ccccc}
  
  & \multicolumn{15}{c}{\textbf{US Equities}} \\
  & \multicolumn{5}{c}{MKT only} & \multicolumn{5}{c}{SMB and MOM only} & \multicolumn{5}{c}{All three factors} \\
  \hline
  & Low & 2 & 3 & 4 & High & Low & 2 & 3 & 4 & High & Low & 2 & 3 & 4 & High \\ 
  \hline
  Small & 0.59 & 0.69 & 0.71 & 0.67 & 0.64 & 0.69 & 0.62 & 0.55 & 0.51 & 0.52 & 0.92 & 0.94 & 0.92 & 0.89 & 0.90 \\ 
  2 & 0.62 & 0.74 & 0.76 & 0.75 & 0.70 & 0.71 & 0.60 & 0.53 & 0.50 & 0.45 & 0.95 & 0.96 & 0.94 & 0.94 & 0.91 \\ 
  3 & 0.64 & 0.77 & 0.82 & 0.81 & 0.73 & 0.63 & 0.53 & 0.46 & 0.42 & 0.38 & 0.92 & 0.94 & 0.94 & 0.94 & 0.91 \\ 
  4 & 0.61 & 0.80 & 0.88 & 0.88 & 0.74 & 0.60 & 0.46 & 0.33 & 0.23 & 0.23 & 0.88 & 0.93 & 0.92 & 0.92 & 0.86 \\ 
  Big & 0.59 & 0.78 & 0.90 & 0.87 & 0.73 & 0.49 & 0.39 & 0.19 & 0.05 & 0.06 & 0.83 & 0.92 & 0.92 & 0.90 & 0.86 \\ 
  \hline
  & \multicolumn{15}{c}{\textbf{EU Equities}} \\
  & \multicolumn{5}{c}{MKT only} & \multicolumn{5}{c}{SMB and MOM only} & \multicolumn{5}{c}{All three factors} \\
  \hline
  Small & 0.75 & 0.78 & 0.77 & 0.73 & 0.70 & 0.51 & 0.37 & 0.27 & 0.23 & 0.19 & 0.96 & 0.94 & 0.93 & 0.92 & 0.92 \\ 
  2 & 0.75 & 0.84 & 0.83 & 0.78 & 0.72 & 0.58 & 0.35 & 0.27 & 0.19 & 0.16 & 0.96 & 0.95 & 0.94 & 0.94 & 0.94 \\ 
  3 & 0.81 & 0.88 & 0.87 & 0.81 & 0.74 & 0.52 & 0.33 & 0.24 & 0.14 & 0.07 & 0.96 & 0.95 & 0.94 & 0.92 & 0.92 \\ 
  4 & 0.80 & 0.92 & 0.90 & 0.84 & 0.74 & 0.52 & 0.31 & 0.17 & 0.08 & 0.02 & 0.93 & 0.95 & 0.92 & 0.91 & 0.91 \\ 
  Big & 0.79 & 0.88 & 0.93 & 0.84 & 0.68 & 0.56 & 0.39 & 0.18 & 0.04 & -0.00 & 0.95 & 0.94 & 0.94 & 0.93 & 0.89 \\ 
   \hline
\end{tabular}
\caption{Adjusted $R^2$ values from the three regressions performed on the 25 portfolios formed on size and momentum, for the markets of an EUR-based investor investing in US Equity and EU Equity. }
\label{R2_EUR}
\end{table}



In Table \ref{Factors_EUR_US} and Table \ref{Factors_EUR_EU} the regression estimates from the three-factor model 
for an EUR-based investor investing in respectively US Equities and EU Equities is shown.

\begin{table}[h]
\centering
\begin{tabular}{lccccc|ccccc}
\hline
  & Low & 2 & 3 & 4 & High & Low & 2 & 3 & 4 & High\\
\hline
& \multicolumn{5}{c|}{$b$} & \multicolumn{5}{c}{$t(b)$} \\
Small & \cellcolor[HTML]{FFFFFF}{1.07} & \cellcolor[HTML]{FFFFFF}{0.91} & \cellcolor[HTML]{FFFFFF}{0.87} & \cellcolor[HTML]{FFFFFF}{0.88} & \cellcolor[HTML]{FFFFFF}{1.02} & 25.84 & 37.34 & 33.43 & 28.21 & 30.30\\
2 & \cellcolor[HTML]{FFFFFF}{1.11} & \cellcolor[HTML]{FFFFFF}{0.98} & \cellcolor[HTML]{FFFFFF}{0.93} & \cellcolor[HTML]{FFFFFF}{0.97} & \cellcolor[HTML]{FFFFFF}{1.12} & 36.12 & 44.86 & 41.46 & 41.04 & 36.17\\
3 & \cellcolor[HTML]{FFFFFF}{1.12} & \cellcolor[HTML]{FFFFFF}{1.00} & \cellcolor[HTML]{FFFFFF}{0.97} & \cellcolor[HTML]{FFFFFF}{0.98} & \cellcolor[HTML]{FFFFFF}{1.11} & 29.37 & 40.08 & 44.23 & 43.97 & 37.97\\
4 & \cellcolor[HTML]{FFFFFF}{1.13} & \cellcolor[HTML]{FFFFFF}{1.01} & \cellcolor[HTML]{FFFFFF}{0.97} & \cellcolor[HTML]{FFFFFF}{0.98} & \cellcolor[HTML]{FFFFFF}{1.11} & 24.21 & 39.99 & 43.52 & 44.20 & 33.64\\
Big & \cellcolor[HTML]{FFFFFF}{1.18} & \cellcolor[HTML]{FFFFFF}{0.96} & \cellcolor[HTML]{FFFFFF}{0.95} & \cellcolor[HTML]{FFFFFF}{0.97} & \cellcolor[HTML]{FFFFFF}{1.07} & 22.38 & 39.66 & 46.52 & 44.87 & 37.24\\
\hline
& \multicolumn{5}{c|}{$s$} & \multicolumn{5}{c}{$t(s)$} \\
Small & \cellcolor[HTML]{FBE723}{1.34} & \cellcolor[HTML]{C2DF23}{1.05} & \cellcolor[HTML]{AADC32}{0.93} & \cellcolor[HTML]{B8DE29}{0.99} & \cellcolor[HTML]{EAE51A}{1.25} & 21.73 & 29.03 & 23.98 & 21.27 & 24.82\\
2 & \cellcolor[HTML]{D0E11C}{1.11} & \cellcolor[HTML]{A8DB34}{0.92} & \cellcolor[HTML]{A2DA37}{0.89} & \cellcolor[HTML]{ADDC30}{0.94} & \cellcolor[HTML]{C8E020}{1.08} & 24.32 & 28.23 & 26.72 & 26.67 & 23.57\\
3 & \cellcolor[HTML]{89D548}{0.76} & \cellcolor[HTML]{73D056}{0.64} & \cellcolor[HTML]{73D056}{0.64} & \cellcolor[HTML]{7FD34E}{0.71} & \cellcolor[HTML]{9BD93C}{0.85} & 13.38 & 17.27 & 19.47 & 21.46 & 19.51\\
4 & \cellcolor[HTML]{65CB5E}{0.56} & \cellcolor[HTML]{42BE71}{0.34} & \cellcolor[HTML]{42BE71}{0.34} & \cellcolor[HTML]{3FBC73}{0.31} & \cellcolor[HTML]{5AC864}{0.49} & 8.10 & 8.96 & 10.13 & 9.54 & 9.86\\
Big & \cellcolor[HTML]{1E9D89}{-0.15} & \cellcolor[HTML]{1E9D89}{-0.15} & \cellcolor[HTML]{1F9F88}{-0.12} & \cellcolor[HTML]{1FA188}{-0.10} & \cellcolor[HTML]{20A486}{-0.04} & -1.96 & -4.26 & -3.92 & -3.26 & -0.90\\
\hline
& \multicolumn{5}{c|}{$m$} & \multicolumn{5}{c}{$t(m)$} \\
Small & \cellcolor[HTML]{2A768E}{-0.69} & \cellcolor[HTML]{1F958B}{-0.25} & \cellcolor[HTML]{1FA187}{-0.08} & \cellcolor[HTML]{26AD81}{0.08} & \cellcolor[HTML]{3BBB75}{0.28} & -18.05 & -11.10 & -3.19 & 2.81 & 9.01\\
2 & \cellcolor[HTML]{2F6C8E}{-0.83} & \cellcolor[HTML]{21908D}{-0.33} & \cellcolor[HTML]{1FA187}{-0.08} & \cellcolor[HTML]{25AC82}{0.06} & \cellcolor[HTML]{40BD72}{0.32} & -29.69 & -16.32 & -3.93 & 2.84 & 11.27\\
3 & \cellcolor[HTML]{2E6E8E}{-0.80} & \cellcolor[HTML]{228D8D}{-0.37} & \cellcolor[HTML]{1E9D89}{-0.15} & \cellcolor[HTML]{25AB82}{0.05} & \cellcolor[HTML]{46C06F}{0.36} & -22.98 & -16.11 & -7.66 & 2.37 & 13.39\\
4 & \cellcolor[HTML]{31688E}{-0.88} & \cellcolor[HTML]{228B8D}{-0.40} & \cellcolor[HTML]{1FA088}{-0.11} & \cellcolor[HTML]{28AE80}{0.10} & \cellcolor[HTML]{4CC26C}{0.40} & -20.73 & -17.46 & -5.63 & 4.81 & 13.11\\
Big & \cellcolor[HTML]{31668E}{-0.91} & \cellcolor[HTML]{24868E}{-0.46} & \cellcolor[HTML]{1F9E89}{-0.14} & \cellcolor[HTML]{2DB27D}{0.15} & \cellcolor[HTML]{4AC16D}{0.39} & -18.79 & -20.64 & -7.51 & 7.64 & 14.96\\
\hline
& \multicolumn{5}{c|}{$R^2$} & \multicolumn{5}{c}{$s(e)$} \\
Small & \cellcolor[HTML]{FFFFFF}{0.92} & \cellcolor[HTML]{FFFFFF}{0.94} & \cellcolor[HTML]{FFFFFF}{0.92} & \cellcolor[HTML]{FFFFFF}{0.89} & \cellcolor[HTML]{FFFFFF}{0.90} & 2.44 & 1.43 & 1.53 & 1.83 & 1.99\\
2 & \cellcolor[HTML]{FFFFFF}{0.95} & \cellcolor[HTML]{FFFFFF}{0.96} & \cellcolor[HTML]{FFFFFF}{0.94} & \cellcolor[HTML]{FFFFFF}{0.94} & \cellcolor[HTML]{FFFFFF}{0.91} & 1.81 & 1.29 & 1.32 & 1.39 & 1.82\\
3 & \cellcolor[HTML]{FFFFFF}{0.92} & \cellcolor[HTML]{FFFFFF}{0.94} & \cellcolor[HTML]{FFFFFF}{0.94} & \cellcolor[HTML]{FFFFFF}{0.94} & \cellcolor[HTML]{FFFFFF}{0.91} & 2.23 & 1.47 & 1.29 & 1.31 & 1.72\\
4 & \cellcolor[HTML]{FFFFFF}{0.88} & \cellcolor[HTML]{FFFFFF}{0.93} & \cellcolor[HTML]{FFFFFF}{0.92} & \cellcolor[HTML]{FFFFFF}{0.92} & \cellcolor[HTML]{FFFFFF}{0.86} & 2.75 & 1.49 & 1.31 & 1.30 & 1.95\\
Big & \cellcolor[HTML]{FFFFFF}{0.83} & \cellcolor[HTML]{FFFFFF}{0.92} & \cellcolor[HTML]{FFFFFF}{0.92} & \cellcolor[HTML]{FFFFFF}{0.90} & \cellcolor[HTML]{FFFFFF}{0.86} & 3.11 & 1.42 & 1.20 & 1.28 & 1.69\\
\hline
\end{tabular}
\caption{3-factor regression of Excess Return on US Equities from an EUR-based investor's perspective. 
Performed on 25 portfolios formed on size and momentum. }
\label{Factors_EUR_US}
\end{table}


\input{R/Output/factors_EUR_EU_colored_format.tex}

From the colored sections, we see the same patterns for the slopes of the size factor (SMB) and the momentum factor (MOM)
as we saw the USD-based investor investing in US Equities. And we again note that the factors are statistically significant in almost all the portfolios.

This truely indicates that the three-factor model is indeed 
also relevant for en EUR-based investor.

Thus our recommendations regarding the asset allocation in Best Pensions' Active Portfolio
will be based on the factors from the three-factor model. 
























\subsection{Preliminary Recommendations}
Since Best Pensions is a EUR-based investor we turn our attention to the 6 explanatory portfolios 
used in the three-factor models for an EUR-based investor investing in US Equities and EU Equities.

In Table \ref{Summary_Factor_PFs} the summary statistics for 
the 6 explanatory portfolios and the european risk-free 
in the period from September 2004 to December 2024 asset are shown.

% latex table generated in R 4.4.1 by xtable 1.8-4 package
% Sun Oct 19 14:23:47 2025
\begin{table}[ht]
\centering
\begin{tabular}{l|@{\hskip 0.7cm} r @{\hskip 0.7cm} |rrrrr}
  \hline
\textbf{Portfolio} & \textbf{Mean} & Min. & Q1 & Median & Q3 & Max. \\ 
  \hline
  European Market Excess & 0.58 & -15.10 & -1.75 & 1.13 & 3.14 & 14.29 \\ 
  US Market Excess  & 0.93 & -13.00 & -1.11 & 1.02 & 3.44 & 14.45 \\ 
  European MOM & 0.77 & -26.15 & -0.56 & 0.95 & 2.56 & 11.37 \\ 
  US MOM & 0.15 & -34.43 & -1.66 & 0.43 & 2.69 & 12.51 \\ 
  European SMB & 0.09 & -5.34 & -1.09 & 0.07 & 1.32 & 5.80 \\ 
  US SMB & 0.00 & -8.73 & -1.79 & -0.03 & 1.63 & 9.33 \\ 
  Risk-free & 0.09 & -0.07 & -0.04 & 0.02 & 0.19 & 0.42 \\ 
   \hline
\end{tabular}
\caption{Summary statistics for the monthly returns of the 6 explanatory portfolios including the European risk-free asset from the period September 2004 to December 2024. 
All values are in percentage points.}
\label{Summary_Factor_PFs}
\end{table}


We note, that the US Market Excess portfolio has the highest mean return of 0.93\% per month,
followed by the European MOM portfolio with a mean return of 0.77\% per month.
The portfolio with the highest median return is the European Market Excess portfolio with a median return of 1.13\% per month.

Thus these could be possible candidates for inclusion in Best Pensions' Active Portfolio.

We also note that both the European SMB portfolio and the US SMB portfolio have very low mean returns of 0.09\% and 0.00\% per month respectively.
These mean returns are lower than the mean return of the european risk-free asset of 0.09\% per month.

Thus we do not recommend including these two portfolios in Best Pensions' Active Portfolio.
\newline 

In Figure \ref{fig:CorPlot_Factor_PFs} the correlations of the monthly returns of the 6 explanatory portfolios are visualized.

We notice, that the Market Excess Portfolios are highly correlated, so including both in the 
Active Portfolio may not provide any diversification benefits.

On the contrary, the US Market Excess portfolio and the European MOM portfolio have a very low correlation.

Thus we would recommend Best Pensions to include the US Market Excess portfolio and the European MOM portfolio 
in their Active Portfolio.

\begin{figure}[H]
    \centering
    \includegraphics[width=0.5\textwidth]{R/Output/MKT_MOM_SMB_corplot.pdf}
    \caption{Correlation plot of monthly returns for the six explanatory portfolios from September 2004 to December 2024.}
    \label{fig:CorPlot_Factor_PFs}
\end{figure}

The expected surplus and covariance matrix in this recommended factor portfolio universe, 
is shown in Table \ref{RecomFactorUniv}.

\begin{table}[ht]
\centering
\begin{tabular}{|l|c|cc|}
  \hline
    &  & \multicolumn{2}{c|}{\textbf{Covariance Matrix}} \\
    & \textbf{Expected Monthly Return} & US Market Excess & European MOM \\
    \hline
    US Market Excess & 0.93 \% & 17.03 & -4.07 \\
    European MOM & 0.77 \% & -4.07 & 12.95 \\
    \hline
\end{tabular}
\caption{Expected monthly returns and covariance matrix for the recommended factor portfolio universe from September 2004 to December 2024. }
\label{RecomFactorUniv}
\end{table}


\subsubsection{Restricting the Active Portfolio to no short-selling}
The Executive Board of Best Pensions has expressed a no short-selling restriction to the Active Portfolio.

This decision prohibits us from investing in the SMB and MOM portfolios, 
since these portfolios are formed by taking both long and short positions in stocks.

Instead we introduce the long-only versions of the MOM and SMB portfolios, which only consists of the 
long portfolios used to form the original SMB and MOM portfolios. 
We denote the long-only MOM portfolios as Tech Stocks and the long-only SMB portfolios as Small Cap.



In Table \ref{Summary_Factor_LongOnly_PFs} the summary statistics for the new long-only investment universe 
are shown. And in figure \ref{fig:CorPlot_Factor_LongOnly_PFs} the correlations of the monthly returns of the 6 long-only portfolios are visualized.

We note, that the portfolio returns are now very correlated. 
This means that the diversification benefits of including multiple portfolios
in the Active Portfolio are very limited.

% latex table generated in R 4.4.1 by xtable 1.8-4 package
% Tue Oct 21 11:36:18 2025
\begin{table}[ht]
\centering
\begin{tabular}{l|@{\hskip 0.7cm} r @{\hskip 0.7cm} |rrrrr}
  \hline
\textbf{Portfolio} & \textbf{Mean} & Min. & Q1 & Median & Q3 & Max. \\ 
  \hline
  European Market & 0.58 & -15.10 & -1.75 & 1.13 & 3.14 & 14.29 \\ 
  US Market  & 0.93 & -13.00 & -1.11 & 1.02 & 3.44 & 14.45 \\ 
  European Tech Stocks & 0.74 & -12.96 & -1.51 & 1.33 & 3.54 & 11.38 \\ 
  US Tech Stocks & 0.99 & -11.08 & -1.79 & 1.17 & 3.91 & 13.54 \\ 
  European Small Cap & 0.63 & -20.21 & -1.75 & 1.48 & 3.29 & 19.48 \\ 
  US Small Cap & 0.91 & -23.44 & -2.54 & 0.97 & 4.32 & 22.71 \\ 
  Risk-free & 0.09 & -0.07 & -0.04 & 0.02 & 0.19 & 0.42 \\ 
   \hline
\end{tabular}
\caption{Summary statistics for the monthly returns of the 6 long-only portfolios including the European risk-free asset from the period September 2004 to December 2024. 
All values, except Risk-free, are excess in percentage points.}
\label{Summary_Factor_LongOnly_PFs}
\end{table}


\begin{figure}[H]
    \centering
    \includegraphics[width=0.5\textwidth]{R/Output/MKT_TECH_SC_corplot.pdf}
    \caption{Correlation plot of monthly returns for the six long-only portfolios from September 2004 to December 2024.}
    \label{fig:CorPlot_Factor_LongOnly_PFs}
\end{figure}




\end{document}
