\subsection{Comparative Analysis of Factor Models}

The Executive Board of Best Pensions has expressed interest in using a more modern factor approach in their asset allocation. 
This approach was originally introduced by Fama and French in 1993, and risk factors
have since become a central component of portfolio theory. 

The market that Fama and French considered in their original paper was the US stock market from 1963 to 1991, thus we 
start by showing that the factor model approach is still relevant in the current market situation, 
and that the approach also applies to the European market.
\newline

The Executive Board has shown particular interest in the market factor, the momentum factor, and the size factor.
The market factor, MKT, suggests that the overall market returns
affects the returns of individual assets. The momentum factor, MOM, suggests that
assets which performed well in the past will continue to perform well in the future.
And the size factor, SMB, suggests that the size of a company affects its returns,
with smaller companies tending to outperform larger companies.
\newline

To ensure that the original analysis is not outdated, we perform a comparative 
analysis of the three factors for the period from September 2004 to December 2024. 
During this analysis we adopt the methodology developed in the original factor analysis by Fama and French.

Moreover, to ensure that the factor approach is relevant to a European investor, 
we analyze the relevancy of the three factors in three different market/investor settings: 

\begin{itemize}
    \item A USD-based investor investing in US Equity.
    \item A EUR-based investor investing in US Equity. 
    \item A EUR-based investor investing in European Equity. 
\end{itemize}

The results of our analysis shows that the three factors are indeed still relevant in all three settings. 
Since Best Pensions is a EUR-based investor investing in possibly both US and European Equities, 
it is indeed suitable for Best Pensions to consider these factors in their asset allocation.
\newline

\subsubsection{Methodology}
To form our proxies for the risk factors, we divide the markets into 6 portfolios based on size and momentum of the stocks, and use the monthly returns of the portfolios.

The size factor,SMB , is calculated by taking the average return of the 3 portfolios containing the samllest firms and 
subtracting the average return of the 3 portfolios containing the largest firms at the end of each month.
The momentum factor, MOM, is calculated by taking the average return of the 2 highest performing portfolios
and subtracting the average return of the 2 worst performing portfolios.
The market factor, RM-RF, is calculated as the excess return of the value-weigted market portfolio over the risk-free rate.
\newline 

We then divide our markets into 25 portfolios based on five size quintiles and five momentum quintiles,
and perform three linear regressions with the following form to each of the portfolios in the 3 market/investor settings:

\begin{align*}
    &\text{1.} \hspace{2cm} R(t)-RF(t) = a + b (RM(t)-RF(t)) + e(t) \\
    &\text{2.} \hspace{2cm} R(t)-RF(t) = a + s SMB(t) + m MOM(t) + e(t) \\
    &\text{3.} \hspace{2cm} R(t)-RF(t) = a + b (RM(t)-RF(t)) + s SMB(t) + m MOM(t) + e(t) \\
\end{align*}

Since the data used in this analysis is from the Kenneth R. French Data Library, 
all of the returns are calculated in USD. 

When considering a EUR-based investors, we therefore need to adjust the returns 
to account for currency loss or gains, by including the return of the USD/EUR-exchange rate. 

\subsubsection{USD-based investor investing in US Equity}
In Table \ref{R2_USD_US} the adjusted $R^2$ values from the three regressions are shown.
From the table we note, that the adjusted $R^2$ values increase significantly when
including all three factors, indicating that the three-factor model explains the variation in bond returns better 
than both the MKT only model and the SMB-MOM model. Thus we concentrate on the three-factor model.
% latex table generated in R 4.4.1 by xtable 1.8-4 package
% Tue Oct 21 07:56:15 2025
\begin{table}[ht]
\centering
\begin{tabular}{lccccc|ccccc|ccccc}
  \hline
    & \multicolumn{5}{c|}{MKT only} & \multicolumn{5}{c|}{SMB and MOM only} & \multicolumn{5}{c}{All three factors} \\ 
 & Low & 2 & 3 & 4 & High & Low & 2 & 3 & 4 & High & Low & 2 & 3 & 4 & High \\ 
  \hline
Small & 0.66 & 0.71 & 0.73 & 0.69 & 0.66 & 0.69 & 0.65 & 0.58 & 0.54 & 0.52 & 0.93 & 0.95 & 0.93 & 0.90 & 0.91 \\ 
  2 & 0.69 & 0.77 & 0.77 & 0.77 & 0.73 & 0.71 & 0.62 & 0.56 & 0.52 & 0.44 & 0.96 & 0.96 & 0.95 & 0.94 & 0.93 \\ 
  3 & 0.70 & 0.81 & 0.84 & 0.83 & 0.76 & 0.65 & 0.55 & 0.49 & 0.45 & 0.36 & 0.93 & 0.95 & 0.95 & 0.94 & 0.93 \\ 
  4 & 0.69 & 0.84 & 0.89 & 0.89 & 0.76 & 0.61 & 0.48 & 0.37 & 0.26 & 0.22 & 0.90 & 0.94 & 0.93 & 0.93 & 0.88 \\ 
  Big & 0.67 & 0.80 & 0.91 & 0.88 & 0.75 & 0.51 & 0.45 & 0.26 & 0.08 & 0.05 & 0.86 & 0.93 & 0.93 & 0.91 & 0.88 \\ 
   \hline
\end{tabular}
\caption{Adjusted $R^2$ values from the three regressions performed on the 25 portfolios formed on size and momentum, for the market of an USD-based investor investing in US Equity. }
\label{R2_USD_US}
\end{table}

\newline

In Table \ref{Factors_USD_US} the regression estimates from the three-factor model 
for a USD-based investor investing in US Equity is shown. 
In the colored sections, we have the slopes of the size factor (SMB) 
and the momentum factor (MOM) for each of the 25 portfolios.

We note that for any fixed momentum quintile,
the slope of the size factor is decreasing as we move from 
the portfolios with the smallest sized assets to the portfolios with 
the largest sized asset.

Similarly, for any fixed size quintile, the slope of the momentum factor is increasing as we move from
the portfolios with the lowest momentum to the portfolios with the highest momentum. 

From the associated t-statistics, we see that in a single testing framework the factors are statistically significant
in almost all the portfolios. 

Moreover, we note that the adjusted R-squared values are all very high, meaning that 
the model explains a lot of the common variation in the portfolio returns.

From this we conclude that the three-factor model is indeed still relevant 
for a USD-based investor investing in US Equity in the current market.

\input{R/Output/factors_USD_US_colored_format.tex}


\newpage
\subsubsection{EUR-based investor investing in US Equity and EU Equity}
An EUR-based investor investing in US equities will both be subject to the returns
of the equities and the returns of the currency exchange rate between USD and EUR.

Moreover, the data from the Kenneth R. French Data Library is all in USD. So the returns on the EU Equities
has been added some exchange rate gain/loss, which we need to remove. 

To account for the exchange rate returns, we transform the USD returns from the Data Library into EUR returns using the 
exchange rate returns from the European Central Bank.
\newline

In Table \ref{R2_EUR} the adjusted $R^2$ values from the three regressions are shown for an EUR-based investor investing in both US Equity and EU Equity.
We again note, that the adjusted $R^2$ values increase significantly when
including all three factors, indicating that the three-factor model explains the common variation in the stock returns better than the other two models. 
We also note, that the adjusted $R^2$ values are close to 1 in most of the portfolios for the three-factor model, yielding that the model is very good at explaining the returns. 
Thus we again concentrate on the three-factor model.
\newline

% latex table generated in R 4.4.1 by xtable 1.8-4 package
% Tue Oct 21 08:41:05 2025
\begin{table}[ht]
\centering
\begin{tabular}{lccccc|ccccc|ccccc}
  
  & \multicolumn{15}{c}{\textbf{US Equities}} \\
  & \multicolumn{5}{c}{MKT only} & \multicolumn{5}{c}{SMB and MOM only} & \multicolumn{5}{c}{All three factors} \\
  \hline
  & Low & 2 & 3 & 4 & High & Low & 2 & 3 & 4 & High & Low & 2 & 3 & 4 & High \\ 
  \hline
  Small & 0.59 & 0.69 & 0.71 & 0.67 & 0.64 & 0.69 & 0.62 & 0.55 & 0.51 & 0.52 & 0.92 & 0.94 & 0.92 & 0.89 & 0.90 \\ 
  2 & 0.62 & 0.74 & 0.76 & 0.75 & 0.70 & 0.71 & 0.60 & 0.53 & 0.50 & 0.45 & 0.95 & 0.96 & 0.94 & 0.94 & 0.91 \\ 
  3 & 0.64 & 0.77 & 0.82 & 0.81 & 0.73 & 0.63 & 0.53 & 0.46 & 0.42 & 0.38 & 0.92 & 0.94 & 0.94 & 0.94 & 0.91 \\ 
  4 & 0.61 & 0.80 & 0.88 & 0.88 & 0.74 & 0.60 & 0.46 & 0.33 & 0.23 & 0.23 & 0.88 & 0.93 & 0.92 & 0.92 & 0.86 \\ 
  Big & 0.59 & 0.78 & 0.90 & 0.87 & 0.73 & 0.49 & 0.39 & 0.19 & 0.05 & 0.06 & 0.83 & 0.92 & 0.92 & 0.90 & 0.86 \\ 
  \hline
  & \multicolumn{15}{c}{\textbf{EU Equities}} \\
  & \multicolumn{5}{c}{MKT only} & \multicolumn{5}{c}{SMB and MOM only} & \multicolumn{5}{c}{All three factors} \\
  \hline
  Small & 0.75 & 0.78 & 0.77 & 0.73 & 0.70 & 0.51 & 0.37 & 0.27 & 0.23 & 0.19 & 0.96 & 0.94 & 0.93 & 0.92 & 0.92 \\ 
  2 & 0.75 & 0.84 & 0.83 & 0.78 & 0.72 & 0.58 & 0.35 & 0.27 & 0.19 & 0.16 & 0.96 & 0.95 & 0.94 & 0.94 & 0.94 \\ 
  3 & 0.81 & 0.88 & 0.87 & 0.81 & 0.74 & 0.52 & 0.33 & 0.24 & 0.14 & 0.07 & 0.96 & 0.95 & 0.94 & 0.92 & 0.92 \\ 
  4 & 0.80 & 0.92 & 0.90 & 0.84 & 0.74 & 0.52 & 0.31 & 0.17 & 0.08 & 0.02 & 0.93 & 0.95 & 0.92 & 0.91 & 0.91 \\ 
  Big & 0.79 & 0.88 & 0.93 & 0.84 & 0.68 & 0.56 & 0.39 & 0.18 & 0.04 & -0.00 & 0.95 & 0.94 & 0.94 & 0.93 & 0.89 \\ 
   \hline
\end{tabular}
\caption{Adjusted $R^2$ values from the three regressions performed on the 25 portfolios formed on size and momentum, for the markets of an EUR-based investor investing in US Equity and EU Equity. }
\label{R2_EUR}
\end{table}



In Table \ref{Factors_EUR_US} and Table \ref{Factors_EUR_EU} the regression estimates from the three-factor model 
for an EUR-based investor investing in respectively US Equities and EU Equities is shown.

\begin{table}[h]
\centering
\begin{tabular}{lccccc|ccccc}
\hline
  & Low & 2 & 3 & 4 & High & Low & 2 & 3 & 4 & High\\
\hline
& \multicolumn{5}{c|}{$b$} & \multicolumn{5}{c}{$t(b)$} \\
Small & \cellcolor[HTML]{FFFFFF}{1.07} & \cellcolor[HTML]{FFFFFF}{0.91} & \cellcolor[HTML]{FFFFFF}{0.87} & \cellcolor[HTML]{FFFFFF}{0.88} & \cellcolor[HTML]{FFFFFF}{1.02} & 25.84 & 37.34 & 33.43 & 28.21 & 30.30\\
2 & \cellcolor[HTML]{FFFFFF}{1.11} & \cellcolor[HTML]{FFFFFF}{0.98} & \cellcolor[HTML]{FFFFFF}{0.93} & \cellcolor[HTML]{FFFFFF}{0.97} & \cellcolor[HTML]{FFFFFF}{1.12} & 36.12 & 44.86 & 41.46 & 41.04 & 36.17\\
3 & \cellcolor[HTML]{FFFFFF}{1.12} & \cellcolor[HTML]{FFFFFF}{1.00} & \cellcolor[HTML]{FFFFFF}{0.97} & \cellcolor[HTML]{FFFFFF}{0.98} & \cellcolor[HTML]{FFFFFF}{1.11} & 29.37 & 40.08 & 44.23 & 43.97 & 37.97\\
4 & \cellcolor[HTML]{FFFFFF}{1.13} & \cellcolor[HTML]{FFFFFF}{1.01} & \cellcolor[HTML]{FFFFFF}{0.97} & \cellcolor[HTML]{FFFFFF}{0.98} & \cellcolor[HTML]{FFFFFF}{1.11} & 24.21 & 39.99 & 43.52 & 44.20 & 33.64\\
Big & \cellcolor[HTML]{FFFFFF}{1.18} & \cellcolor[HTML]{FFFFFF}{0.96} & \cellcolor[HTML]{FFFFFF}{0.95} & \cellcolor[HTML]{FFFFFF}{0.97} & \cellcolor[HTML]{FFFFFF}{1.07} & 22.38 & 39.66 & 46.52 & 44.87 & 37.24\\
\hline
& \multicolumn{5}{c|}{$s$} & \multicolumn{5}{c}{$t(s)$} \\
Small & \cellcolor[HTML]{FBE723}{1.34} & \cellcolor[HTML]{C2DF23}{1.05} & \cellcolor[HTML]{AADC32}{0.93} & \cellcolor[HTML]{B8DE29}{0.99} & \cellcolor[HTML]{EAE51A}{1.25} & 21.73 & 29.03 & 23.98 & 21.27 & 24.82\\
2 & \cellcolor[HTML]{D0E11C}{1.11} & \cellcolor[HTML]{A8DB34}{0.92} & \cellcolor[HTML]{A2DA37}{0.89} & \cellcolor[HTML]{ADDC30}{0.94} & \cellcolor[HTML]{C8E020}{1.08} & 24.32 & 28.23 & 26.72 & 26.67 & 23.57\\
3 & \cellcolor[HTML]{89D548}{0.76} & \cellcolor[HTML]{73D056}{0.64} & \cellcolor[HTML]{73D056}{0.64} & \cellcolor[HTML]{7FD34E}{0.71} & \cellcolor[HTML]{9BD93C}{0.85} & 13.38 & 17.27 & 19.47 & 21.46 & 19.51\\
4 & \cellcolor[HTML]{65CB5E}{0.56} & \cellcolor[HTML]{42BE71}{0.34} & \cellcolor[HTML]{42BE71}{0.34} & \cellcolor[HTML]{3FBC73}{0.31} & \cellcolor[HTML]{5AC864}{0.49} & 8.10 & 8.96 & 10.13 & 9.54 & 9.86\\
Big & \cellcolor[HTML]{1E9D89}{-0.15} & \cellcolor[HTML]{1E9D89}{-0.15} & \cellcolor[HTML]{1F9F88}{-0.12} & \cellcolor[HTML]{1FA188}{-0.10} & \cellcolor[HTML]{20A486}{-0.04} & -1.96 & -4.26 & -3.92 & -3.26 & -0.90\\
\hline
& \multicolumn{5}{c|}{$m$} & \multicolumn{5}{c}{$t(m)$} \\
Small & \cellcolor[HTML]{2A768E}{-0.69} & \cellcolor[HTML]{1F958B}{-0.25} & \cellcolor[HTML]{1FA187}{-0.08} & \cellcolor[HTML]{26AD81}{0.08} & \cellcolor[HTML]{3BBB75}{0.28} & -18.05 & -11.10 & -3.19 & 2.81 & 9.01\\
2 & \cellcolor[HTML]{2F6C8E}{-0.83} & \cellcolor[HTML]{21908D}{-0.33} & \cellcolor[HTML]{1FA187}{-0.08} & \cellcolor[HTML]{25AC82}{0.06} & \cellcolor[HTML]{40BD72}{0.32} & -29.69 & -16.32 & -3.93 & 2.84 & 11.27\\
3 & \cellcolor[HTML]{2E6E8E}{-0.80} & \cellcolor[HTML]{228D8D}{-0.37} & \cellcolor[HTML]{1E9D89}{-0.15} & \cellcolor[HTML]{25AB82}{0.05} & \cellcolor[HTML]{46C06F}{0.36} & -22.98 & -16.11 & -7.66 & 2.37 & 13.39\\
4 & \cellcolor[HTML]{31688E}{-0.88} & \cellcolor[HTML]{228B8D}{-0.40} & \cellcolor[HTML]{1FA088}{-0.11} & \cellcolor[HTML]{28AE80}{0.10} & \cellcolor[HTML]{4CC26C}{0.40} & -20.73 & -17.46 & -5.63 & 4.81 & 13.11\\
Big & \cellcolor[HTML]{31668E}{-0.91} & \cellcolor[HTML]{24868E}{-0.46} & \cellcolor[HTML]{1F9E89}{-0.14} & \cellcolor[HTML]{2DB27D}{0.15} & \cellcolor[HTML]{4AC16D}{0.39} & -18.79 & -20.64 & -7.51 & 7.64 & 14.96\\
\hline
& \multicolumn{5}{c|}{$R^2$} & \multicolumn{5}{c}{$s(e)$} \\
Small & \cellcolor[HTML]{FFFFFF}{0.92} & \cellcolor[HTML]{FFFFFF}{0.94} & \cellcolor[HTML]{FFFFFF}{0.92} & \cellcolor[HTML]{FFFFFF}{0.89} & \cellcolor[HTML]{FFFFFF}{0.90} & 2.44 & 1.43 & 1.53 & 1.83 & 1.99\\
2 & \cellcolor[HTML]{FFFFFF}{0.95} & \cellcolor[HTML]{FFFFFF}{0.96} & \cellcolor[HTML]{FFFFFF}{0.94} & \cellcolor[HTML]{FFFFFF}{0.94} & \cellcolor[HTML]{FFFFFF}{0.91} & 1.81 & 1.29 & 1.32 & 1.39 & 1.82\\
3 & \cellcolor[HTML]{FFFFFF}{0.92} & \cellcolor[HTML]{FFFFFF}{0.94} & \cellcolor[HTML]{FFFFFF}{0.94} & \cellcolor[HTML]{FFFFFF}{0.94} & \cellcolor[HTML]{FFFFFF}{0.91} & 2.23 & 1.47 & 1.29 & 1.31 & 1.72\\
4 & \cellcolor[HTML]{FFFFFF}{0.88} & \cellcolor[HTML]{FFFFFF}{0.93} & \cellcolor[HTML]{FFFFFF}{0.92} & \cellcolor[HTML]{FFFFFF}{0.92} & \cellcolor[HTML]{FFFFFF}{0.86} & 2.75 & 1.49 & 1.31 & 1.30 & 1.95\\
Big & \cellcolor[HTML]{FFFFFF}{0.83} & \cellcolor[HTML]{FFFFFF}{0.92} & \cellcolor[HTML]{FFFFFF}{0.92} & \cellcolor[HTML]{FFFFFF}{0.90} & \cellcolor[HTML]{FFFFFF}{0.86} & 3.11 & 1.42 & 1.20 & 1.28 & 1.69\\
\hline
\end{tabular}
\caption{3-factor regression of Excess Return on US Equities from an EUR-based investor's perspective. 
Performed on 25 portfolios formed on size and momentum. }
\label{Factors_EUR_US}
\end{table}


\input{R/Output/factors_EUR_EU_colored_format.tex}

From the colored sections, we see the same patterns for the slopes of the size factor (SMB) and the momentum factor (MOM)
as we saw the USD-based investor investing in US Equities. And we again note that the factors are statistically significant in almost all the portfolios.

This truely indicates that the three-factor model is indeed 
also relevant for en EUR-based investor.

Thus our recommendations regarding the asset allocation in Best Pensions' Active Portfolio
will be based on the factors from the three-factor model. 






















