\subsection{Preliminary Recommendations}
Since Best Pensions is a EUR-based investor we turn our attention to the 6 explanatory portfolios 
used in the three-factor models for an EUR-based investor investing in US Equities and EU Equities.

In Table \ref{Summary_Factor_PFs} the summary statistics for 
the 6 explanatory portfolios and the european risk-free 
in the period from September 2004 to December 2024 asset are shown.

% latex table generated in R 4.4.1 by xtable 1.8-4 package
% Sun Oct 19 14:23:47 2025
\begin{table}[ht]
\centering
\begin{tabular}{l|@{\hskip 0.7cm} r @{\hskip 0.7cm} |rrrrr}
  \hline
\textbf{Portfolio} & \textbf{Mean} & Min. & Q1 & Median & Q3 & Max. \\ 
  \hline
  European Market Excess & 0.58 & -15.10 & -1.75 & 1.13 & 3.14 & 14.29 \\ 
  US Market Excess  & 0.93 & -13.00 & -1.11 & 1.02 & 3.44 & 14.45 \\ 
  European MOM & 0.77 & -26.15 & -0.56 & 0.95 & 2.56 & 11.37 \\ 
  US MOM & 0.15 & -34.43 & -1.66 & 0.43 & 2.69 & 12.51 \\ 
  European SMB & 0.09 & -5.34 & -1.09 & 0.07 & 1.32 & 5.80 \\ 
  US SMB & 0.00 & -8.73 & -1.79 & -0.03 & 1.63 & 9.33 \\ 
  Risk-free & 0.09 & -0.07 & -0.04 & 0.02 & 0.19 & 0.42 \\ 
   \hline
\end{tabular}
\caption{Summary statistics for the monthly returns of the 6 explanatory portfolios including the European risk-free asset from the period September 2004 to December 2024. 
All values are in percentage points.}
\label{Summary_Factor_PFs}
\end{table}


We note, that the US Market Excess portfolio has the highest mean return of 0.93\% per month,
followed by the European MOM portfolio with a mean return of 0.77\% per month.
The portfolio with the highest median return is the European Market Excess portfolio with a median return of 1.13\% per month.

Thus these could be possible candidates for inclusion in Best Pensions' Active Portfolio.

We also note that both the European SMB portfolio and the US SMB portfolio have very low mean returns of 0.09\% and 0.00\% per month respectively.
These mean returns are lower than the mean return of the european risk-free asset of 0.09\% per month.

Thus we do not recommend including these two portfolios in Best Pensions' Active Portfolio.
\newline 

In Figure \ref{fig:CorPlot_Factor_PFs} the correlations of the monthly returns of the 6 explanatory portfolios are visualized.

We notice, that the Market Excess Portfolios are highly correlated, so including both in the 
Active Portfolio may not provide any diversification benefits.

On the contrary, the US Market Excess portfolio and the European MOM portfolio have a very low correlation.

Thus we would recommend Best Pensions to include the US Market Excess portfolio and the European MOM portfolio 
in their Active Portfolio.

\begin{figure}[H]
    \centering
    \includegraphics[width=0.5\textwidth]{R/Output/MKT_MOM_SMB_corplot.pdf}
    \caption{Correlation plot of monthly returns for the six explanatory portfolios from September 2004 to December 2024.}
    \label{fig:CorPlot_Factor_PFs}
\end{figure}

The expected surplus and covariance matrix in this recommended factor portfolio universe, 
is shown in Table \ref{RecomFactorUniv}.

\begin{table}[ht]
\centering
\begin{tabular}{|l|c|cc|}
  \hline
    &  & \multicolumn{2}{c|}{\textbf{Covariance Matrix}} \\
    & \textbf{Expected Monthly Return} & US Market Excess & European MOM \\
    \hline
    US Market Excess & 0.93 \% & 17.03 & -4.07 \\
    European MOM & 0.77 \% & -4.07 & 12.95 \\
    \hline
\end{tabular}
\caption{Expected monthly returns and covariance matrix for the recommended factor portfolio universe from September 2004 to December 2024. }
\label{RecomFactorUniv}
\end{table}


\subsubsection{Restricting the Active Portfolio to no short-selling}
The Executive Board of Best Pensions has expressed a no short-selling restriction to the Active Portfolio.

This decision prohibits us from investing in the SMB and MOM portfolios, 
since these portfolios are formed by taking both long and short positions in stocks.

Instead we introduce the long-only versions of the MOM and SMB portfolios, which only consists of the 
long portfolios used to form the original SMB and MOM portfolios. 
We denote the long-only MOM portfolios as Tech Stocks and the long-only SMB portfolios as Small Cap.



In Table \ref{Summary_Factor_LongOnly_PFs} the summary statistics for the new long-only investment universe 
are shown. And in figure \ref{fig:CorPlot_Factor_LongOnly_PFs} the correlations of the monthly returns of the 6 long-only portfolios are visualized.

We note, that the portfolio returns are now very correlated. 
This means that the diversification benefits of including multiple portfolios
in the Active Portfolio are very limited.

% latex table generated in R 4.4.1 by xtable 1.8-4 package
% Tue Oct 21 11:36:18 2025
\begin{table}[ht]
\centering
\begin{tabular}{l|@{\hskip 0.7cm} r @{\hskip 0.7cm} |rrrrr}
  \hline
\textbf{Portfolio} & \textbf{Mean} & Min. & Q1 & Median & Q3 & Max. \\ 
  \hline
  European Market & 0.58 & -15.10 & -1.75 & 1.13 & 3.14 & 14.29 \\ 
  US Market  & 0.93 & -13.00 & -1.11 & 1.02 & 3.44 & 14.45 \\ 
  European Tech Stocks & 0.74 & -12.96 & -1.51 & 1.33 & 3.54 & 11.38 \\ 
  US Tech Stocks & 0.99 & -11.08 & -1.79 & 1.17 & 3.91 & 13.54 \\ 
  European Small Cap & 0.63 & -20.21 & -1.75 & 1.48 & 3.29 & 19.48 \\ 
  US Small Cap & 0.91 & -23.44 & -2.54 & 0.97 & 4.32 & 22.71 \\ 
  Risk-free & 0.09 & -0.07 & -0.04 & 0.02 & 0.19 & 0.42 \\ 
   \hline
\end{tabular}
\caption{Summary statistics for the monthly returns of the 6 long-only portfolios including the European risk-free asset from the period September 2004 to December 2024. 
All values, except Risk-free, are excess in percentage points.}
\label{Summary_Factor_LongOnly_PFs}
\end{table}


\begin{figure}[H]
    \centering
    \includegraphics[width=0.5\textwidth]{R/Output/MKT_TECH_SC_corplot.pdf}
    \caption{Correlation plot of monthly returns for the six long-only portfolios from September 2004 to December 2024.}
    \label{fig:CorPlot_Factor_LongOnly_PFs}
\end{figure}


